\documentclass[12pt,a4paper, titlepage]{article}
\usepackage{amsmath}
\usepackage{graphicx}
\usepackage{float}
\usepackage{wrapfig}
\usepackage[russian]{babel}
\usepackage[utf8]{inputenc}



\title{sss}
\date{2021\\Март}
\author{Петраков Иван\\\ МФТИ\\\\\\\\\\\\\\\\\\\\}
\hoffset = 0pt
\voffset = 0pt
\textheight = 700pt
\topmargin = 0pt
\headheight = 0pt
\headsep = 0pt
\marginparwidth = 0pt
\oddsidemargin = 0pt
\textwidth = 450pt

\begin{document}
Time-optimal reorientation of a rigid spacecraft.
\\
\\
Hello everyone, thank you for coming. As I'm sure you are studying in a space-related department in one way or another, you probably know something about spacecraft reorientation. But may I ask the question: which particular manoeuvre, resulting in a change of the spacecraft's angle, has the least time expenditure? The search for such manoeuvres, called time-optimal, has been done by many scientists of this and last century, and today we will talk with you about the basics of time-optimal reorientation of a rigid spacecraft, skipping complicated mathematical calculations. Namely, today we will talk about the formulation of the problem, one of the methods of solving it and the results obtained with the software code. If you have any questions, I will be happy to answer them at the end of the presentation.
\\
\\
Let's start with the wording of the problem. Figure shows the formulation of the problem itself, where x, y, z are interial axes, b1, b2, b3 are body-fixed axes, $\Omega$ is the vector of angular velocity, and $\tau$ are independent control torques about body-fixed axis. From the course of theoretical mechanics, we know the Euler equations, which, with a certain replacement of variables, reduce to a fairly simple form, where the derivative of the dimensionless angular velocity is equal to the dimensionless control. Then the problem can be posed as follows: given that u is some control function on time, giving the rotation, translate the system described by initial quaternions (of which there are 4 in total) into the system described by finite quaternions, minimizing the cost function equal to the integral from 0 to finite time on dt. As we see, the formulation of the problem is simple enough, but the simpler the problem, the more complicated the solution, but not in our case.
\\
\\
Let's move on to a possible solution to this problem. One of the most primitive solutions is a solution based on the bang-bang principle. It states that by using the maximum and minimum possible control components, which in our case are defined on the interval from -1 to 1, we can minimise the cost function. Next, you may ask, where did this solution come from? In fact, the optimal controls should minimize the Hamiltonian of the system, corresponding to the Miminum Principle, and then there are rather complicated calculations, which we have agreed to omit. Thus, we see that the solution, as well as the formulation of the problem, is quite simple.
\\
\\
Let's discuss the last part of my presentation - the implementation of the task as program code. Using Python3, I set the initial and final conditions of the problem. Then, I implemented the solution as program code. Finally, I plotted the angular velocity versus angle, where the blue dot is what we want to achieve and the red line is the path in the phase space that specifies the time-optimal maneuver. In this way, it is possible to obtain solutions for absolutely any initial and final conditions of the problem.
\\
\\
Let's summarise what we have discussed today. We talked about the problem statement, which in short sounds like "minimizing the cost function under given conditions", discussed a bit about the theoretical solution to this problem, which is based on the bang-bang principle, and then talked about the software implementation of the solution by looking at the graph I obtained. In fact, in solving this problem I referred to an article from the last century. You may ask: Is this problem relevant now? The answer is yes. The fact is that the solution considered is not accurate enough in such an ultra-precise field of science, so there is still a search for a more accurate solution. Perhaps some of you will look into this for the sake of advancing science.
\end{document}